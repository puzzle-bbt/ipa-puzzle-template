\documentclass[12pt]{report}

\usepackage[utf8]{inputenc}
\usepackage{lmodern}
\usepackage{amsmath}
\usepackage[ngerman]{babel}
\usepackage{enumitem}% http://ctan.org/pkg/enumitem
\usepackage[ngerman]{translator} % Weitere Übersetzungen
\usepackage{inconsolata}
\usepackage{graphicx}
\usepackage{layout}
\usepackage{listings}
\usepackage{fancyhdr}
\usepackage{makecell}
\usepackage{titlesec}
\usepackage{roboto}
\usepackage{sectsty}
\usepackage{lastpage}
\usepackage{longtable}
\usepackage{tabularx}
\usepackage[table]{xcolor}
\usepackage{tikz}
\usepackage{array}
\usepackage{lipsum}
\usepackage{multirow}
\usepackage{float}
\usepackage{caption}
\usepackage{hyperref}
\usepackage{geometry}            % Dient zur Interface dimensionierung
\usepackage{pdflscape}           % bessere Lesbarkeit des PDF Dokuments
\usepackage{typearea}  % loaded automatically if using KOMA-Script class
\usepackage{pgfgantt}
\usepackage{wrapfig}
\usepackage{soulutf8}
\usepackage{pdfpages}
\usepackage{titletoc}
\usepackage{tocloft}
\usepackage{atbegshi,ifthen}

\lstset{language=ruby}

\hypersetup{%
    colorlinks=true,
    filecolor=black,
    linkcolor=black,
    urlcolor=blue,
    citecolor=black,
    pdfborderstyle={/S/U/W 0}% border style will be underline of width 1pt
}

\def\checkmark{\tikz\fill[scale=0.4](0,.35) -- (.25,0) -- (1,.7) -- (.25,.15) -- cycle;}
\newcounter{highlight}[page]
\newcommand{\tikzhighlightanchor}[1]{\ensuremath{\vcenter{\hbox{\tikz[remember picture, overlay]{\coordinate (#1 highlight \arabic{highlight});}}}}}
\newcommand{\bh}[0]{\stepcounter{highlight}\tikzhighlightanchor{begin}}
\newcommand{\eh}[0]{\tikzhighlightanchor{end}}
\AtBeginShipout{\AtBeginShipoutUpperLeft{\ifthenelse{\value{highlight} > 0}{\tikz[remember picture, overlay]{\foreach \stroke in {1,...,\arabic{highlight}} \draw[highlighter] (begin highlight \stroke) -- (end highlight \stroke);}}{}}}


%Tabellen Caption links orientiert
\captionsetup[table]{skip=0pt,singlelinecheck=off}

% Redefine the \chapter command to have the chapter number followed by a space and then the chapter title
\makeatletter
\def\@makechapterhead#1{%
  \vspace*{0\p@}%
  {\parindent \z@ \raggedright \normalfont
    \interlinepenalty\@M
    \Huge\bfseries \ifnum \c@secnumdepth >\m@ne
    \thechapter\ \fi % Add a space after \thechapter
    \Huge\bfseries #1\par\nobreak
    \vskip 20\p@
  }}
\makeatother

% Adjust the spacing for chapter entries in the table of contents
\titlecontents{chapter}
  [0em] % left margin
  {\vspace{5pt}} % above code
  {\bfseries\thecontentslabel\quad} % numbered format
  {\bfseries} % unnumbered format
  {\hfill\contentspage} % filler and page number format
  [\vspace{5pt}] % below code

% Remove space above the chapter heading in the table of contents
\renewcommand{\cftbeforechapskip}{0pt}

% Remove space above the entries in the list of tables
\setlength{\cftbeforelottitleskip}{0pt}

% Remove space above the entries in the list of figures

%Tabellen- & Abbildungsverzeichnis als Section und in Inhaltsverzeichnis

%Weitere Spalten typen
\newcolumntype{L}[1]{>{\raggedright\let\newline\\\arraybackslash\hspace{0pt}}m{#1}}
\newcolumntype{C}[1]{>{\centering\let\newline\\\arraybackslash\hspace{0pt}}m{#1}}
\newcolumntype{R}[1]{>{\raggedleft\let\newline\\\arraybackslash\hspace{0pt}}m{#1}}

\newcommand{\changefont}{%
    \fontsize{9}{11}\selectfont
}
%Seitenformatierung anpassen
\headsep=15pt
\textwidth=400pt
\textheight=650pt
\oddsidemargin=25pt
\footskip=20pt
\fancypagestyle{plain}{}
\pagestyle{fancy}
\voffset=0pt
\setlength{\parindent}{0pt}

%Header höhe anpassen
\setlength\headheight{33pt}
%Bilderpfad
\graphicspath{{bilder/}}
%Header und Footer
\fancyhf{}
\cfoot{\today}
\rfoot{\makebox[15cm][r]{Seite \thepage\hspace{1pt} von \pageref*{LastPage}}} % Fixed width footer
\lfoot{\makebox[15cm][l]{Version 1.0}} % Fixed width footer
\chead{\changefont Hitobito: Einführung von PostgreSQL Fulltext Search}
\lhead{\includegraphics[width=2.9cm]{puzzle_tagline_blau}}
\rhead{Niklas Jäggi}

%Weiteres

\definecolor{puzzleblue}{RGB}{30,90,150}
\definecolor{burntorange}{rgb}{1, 0.33, 0.0}
\definecolor{darkmagenta}{rgb}{0.55, 0.0, 0.55}
\definecolor{electricpurple}{rgb}{0.75, 0.0, 1.0}
\definecolor{feldgrau}{HTML}{009900}
\definecolor{ferngreen}{HTML}{99cc99}
\definecolor{darkred}{rgb}{0.8, 0.0, 0.0}
\definecolor{darkblue}{rgb}{0.0, 0.0, 0.55}
\author{Niklas Jäggi}
\title{IPA}

% Gantt

\newcommand{\sollD}[4]{\ganttbar{#1}{#3}{#4}\ganttbar[inline, bar/.append style={bottom color=puzzleblue, top color=puzzleblue!70}]{#2}{#3}{#4}}
\newcommand{\istD}[3]{\ganttbar[inline, bar top shift=-0.01,bar/.append style={bottom color=puzzleblue!80, top color=puzzleblue!30}]{#1}{#2}{#3}}

\newcommand{\sollI}[4]{\ganttbar{#1}{#3}{#4}\ganttbar[inline, bar/.append style={bottom color=darkred!90, top color=darkred!50}]{#2}{#3}{#4}}
\newcommand{\istI}[3]{\ganttbar[inline, bar top shift=-0.01,bar/.append style={bottom color=red!80, top color=red!30}]{#1}{#2}{#3}}

\newcommand{\sollK}[4]{\ganttbar{#1}{#3}{#4}\ganttbar[inline, bar/.append style={bottom color=feldgrau, top color=feldgrau!50}]{#2}{#3}{#4}}
\newcommand{\istK}[3]{\ganttbar[inline, bar top shift=-0.01,bar/.append style={bottom color=ferngreen, top color=ferngreen!50}]{#1}{#2}{#3}}

\newcommand{\sollR}[4]{\ganttbar{#1}{#3}{#4}\ganttbar[inline, bar/.append style={bottom color=darkmagenta!80, top color=darkmagenta!50}]{#2}{#3}{#4}}
\newcommand{\istR}[3]{\ganttbar[inline, bar top shift=-0.01,bar/.append style={bottom color=electricpurple!80, top color=electricpurple!30}]{#1}{#2}{#3}}

\newcommand{\sollE}[4]{\ganttbar{#1}{#3}{#4}\ganttbar[inline, bar/.append style={bottom color=burntorange, top color=burntorange!50}]{#2}{#3}{#4}}
\newcommand{\istE}[3]{\ganttbar[inline, bar top shift=-0.01,bar/.append style={bottom color=orange, top color=orange!50}]{#1}{#2}{#3}}

\definecolor{mygreen}{rgb}{0,0.6,0}
\definecolor{mygray}{rgb}{0.5,0.5,0.5}
\definecolor{mymauve}{rgb}{0.58,0,0.82}
\lstset{ 
  backgroundcolor=\color{gray!30},   % choose the background color; you must add \usepackage{color} or \usepackage{xcolor}; should come as last argument
  basicstyle=\scriptsize\ttfamily,        % the size of the fonts that are used for the code
  breakatwhitespace=true,         % sets if automatic breaks should only happen at whitespace
  breaklines=true,                 % sets automatic line breaking
  captionpos=b,                    % sets the caption-position to bottom
  commentstyle=\color{mygreen},    % comment style
  deletekeywords={...},            % if you want to delete keywords from the given language
  escapeinside={\%*}{*)},          % if you want to add LaTeX within your code
  extendedchars=true,              % lets you use non-ASCII characters; for 8-bits encodings only, does not work with UTF-8
  frame=single,	                   % adds a frame around the code
  keepspaces=true,                 % keeps spaces in text, useful for keeping indentation of code (possibly needs columns=flexible)
  keywordstyle=\color{blue},       % keyword style
  language=Ruby,                 % the language of the code
  morekeywords={*,rescue, it},            % if you want to add more keywords to the set
  numbers=left,                    % where to put the line-numbers; possible values are (none, left, right)
  numbersep=5pt,                   % how far the line-numbers are from the code
  numberstyle=\tiny\color{mygray}, % the style that is used for the line-numbers
  rulecolor=\color{black},         % if not set, the frame-color may be changed on line-breaks within not-black text (e.g. comments (green here))
  showspaces=false,                % show spaces everywhere adding particular underscores; it overrides 'showstringspaces'
  showstringspaces=false,          % underline spaces within strings only
  showtabs=false,                  % show tabs within strings adding particular underscores
  stepnumber=1,                    % the step between two line-numbers. If it's 1, each line will be numbered
  stringstyle=\color{mymauve},     % string literal style
  tabsize=2,	                   % sets default tabsize to 2 spaces
  title=\lstname                   % show the filename of files included with \lstinputlisting; also try caption instead of title
}







\title{TODO:Titel einfügen}

\author{TODO:Autor einfügen}

\date{TODO:Datum eintragen}

\begin{document}
{
  % Use \raggedright to prevent hyphenation and word splitting
  \raggedright
  % Adjust spacing between paragraphs
  \setlength{\parskip}{\baselineskip}
  \begin{titlepage}
    \Huge IPA [Name] - Puzzle ITC\normalsize
  \bigbreak
  \begin{table}[h!]
    \rowcolors{2}{puzzleblue!30}{white}
      \begin{tabular}{|L{0.4\textwidth}|L{0.5\textwidth}|}
          \hline
          \rowcolor{puzzleblue} \multicolumn{2}{|l|}{\color{white}\textbf{IPA-Daten und beteiligte Personen}} \\[12pt]
          \hline
          Firma & Puzzle ITC \\
          \hline
          Abteilung & \\
          \hline
          Berufsschule & \\
          \hline
          Valid Experte & \\
          \hline
          Hauptexpertin & \\
          \hline
          Nebenexperte & \\
          \hline
          Verantwortliche Fachkraft & \\
          \hline
          Zusätzliche Verantwortliche Fachkraft & \\
          \hline
          Autor & \\
          \hline
          Abgabedatum & \\
          \hline
          Fachrichtung & \\
          \hline
          Projektvorgehensmodell & SCRUM \\
          \hline
          Jahrgang & \\
          \hline
        \end{tabular}
        \caption{IPA Daten}
  \end{table}
\end{titlepage}

  \part[Ablauf Organisation und Umfeld]{Ablauf Organisation und Umfeld
        \begin{center}
            \begin{minipage}[c]{10.7cm}
            \small TODO:Name der IPA einfügen \\
            Autor: TODO: Autor einfügen
            \end{minipage}
        \end{center}
    }
\chapter{Kurzfassung des IPA-Berichtes}
\section{Kurze Ausgangssituation}

\section{Umsetzung}

\section{Ergebnis}

\tableofcontents

\chapter{Aufgabenstellung}

\section{Titel der Arbeit}

\section{Ausgangslage}

\section{Detaillierte Aufgabenstellung}

\section{Mittel und Methoden}

\subsection{Technologie und Plattform}

\subsection{Entwicklungsumgebung}

\subsection{Textverarbeitung und Diagramme}

\subsection{Projektmethode}

\subsection{Konventionen}

\section{Vorkenntnisse}

\section{Vorarbeiten}

\section{Neue Lerninhalte}

\section{Arbeiten in den letzten 6 Monaten}

\chapter{Firmenstandards}

\chapter{IPA Schutzbedarfanalyse}

\chapter{Organisation der IPA}

\section{Arbeitsplatz}
  
\section{Datensicherung der IPA}

\section{Änderungskontrolle}

\begin{table}[H]
    \begin{tabular}{|L{0.15\textwidth}|L{0.15\textwidth}|L{0.18\textwidth}|L{0.41\textwidth}|}
        \hline
        \rowcolor{puzzleblue}\color{white}Version &\color{white} Datum & \color{white}Name & \color{white}Beschreibung \\
        \hline
        Vorlage & TODO:Datum einfügen & TODO:Autor einfügen & TODO: Beschreibung einfügen \\
        \hline
    \end{tabular}
    \caption{Änderungsprotokoll}
\end{table}

\chapter{Detailliertes Projektvorgehen}

\section{Projektvorgehen}

\subsection{Projektmethode}

\section{Projektaufbauorganisation}
\subsection{Projektrollen SCRUM}

\begin{table}[H]
    \begin{tabular}{|L{0.3\textwidth}|L{0.65\textwidth}|}
        \hline
        \rowcolor{puzzleblue} \multicolumn{2}{|l|}{\color{white}\textbf{Rollenbeschreibung}} \\[12pt]
        \hline
        \textbf{TODO: Rollennamen} & TODO: Beschreibung der Rolle \\
        \hline
    \end{tabular}
    \caption{Rollenbeschreibung SCRUM}
\end{table}

\subsection{Projektrollen IPA}

\begin{table}[H]
    \begin{tabular}{|L{0.3\textwidth}|L{0.65\textwidth}|}
        \hline
        \rowcolor{puzzleblue} \multicolumn{2}{|l|}{\color{white}\textbf{Rollenbeschreibung}} \\[12pt]
        \hline
        \textbf{TODO: Rollennamen} & TODO: Beschreibung der Rolle \\
        \hline
    \end{tabular}
    \caption{Rollenbeschreibung IPA}
\end{table}

\subsection{Rollenverteilung}

\begin{table}[H]
    \begin{tabular}{|L{0.3\textwidth}|L{0.65\textwidth}|}
        \hline
        \rowcolor{puzzleblue}\color{white}Rolle &\color{white} Verantwortliche Person \\
        \hline
        \textbf{TODO: Rollennamen} & TODO: Person erfassen \\
        \hline
    \end{tabular}
    \caption{Rollenverteiling IPA}
\end{table}

\storeareas\riskvalues
\KOMAoptions{paper=a3, paper=landscape, DIV=current}
\areaset
  {\dimexpr\the\paperwidth-1cm\relax}
  {\dimexpr\the\paperheight-5.5cm\relax}
\recalctypearea

\chapter{Risikoanalyse}
\begin{table}[H]
  \begin{tabular}{ |C{0.01\textwidth}|C{0.1\textwidth}|C{0.1\textwidth}|C{0.02\textwidth}|C{0.02\textwidth}|C{0.03\textwidth}|C{0.1\textwidth}|C{0.2\textwidth}|C{0.02\textwidth}|C{0.02\textwidth}|C{0.03\textwidth}|C{0.1\textwidth}| }
      \hline
      \multirow{2}*{Nr} & \multirow{2}*{Risikobeschreibung} & \multirow{2}*{Auswirkung} & \multicolumn{4}{|l|}{Vor Massnahme}& \multirow{2}*{Massnahmen} & \multicolumn{4}{|l|}{Nach Massnahme} \\
      \cline{4-7} \cline{9-12}&&& W & S & Risiko & Handlungsweise &&  W & S & Risiko & Handlungsweise \\
      \hline 
      1 & TODO: Name & TODO: Beschreibung & TODO: Auswirkung & TODO: Auswirkung & \cellcolor{yellow}TODO: Risiko & TODO: Handlungsweise 
      & TODO: Massnahme & TODO: Auswirkung & TODO: Auswirkung & \cellcolor{yellow}TODO: Risiko & TODO: Handlungsweise \\
      \hline
  \end{tabular}
  \caption{Risikoanalyse}
\end{table}

\textbf{Schadensausmass:} \\
S1 = führt zu keiner Abwertung \\
S2 = geringe Abwertung bis 1,0 Notenpunkte \\
S3 = hohe Abwertung über 1,0 Notenpunkte \\
S4 = führt zu Nichtbestehen \\

\textbf{Eintrittswahrscheinlichkeit:} \\
W1 = unvorstellbar \\
W2 = unwahrscheinlich \\
W3 = eher vorstellbar \\
W4 = vorstellbar \\
W5 = Eintreffen hoch \\

\restoregeometry
\riskvalues





\chapter{Zeitplan}

\section{Erläuterung zum Zeitplan}
% TODO: Beschreibe den Zeitplan, um eventuelle Unklarheiten zu vermeiden

\newpage
\storeareas\zeitplan
\KOMAoptions{paper=a3, paper=landscape, DIV=current}
\areaset
  {\dimexpr\the\paperwidth-1cm\relax}% calculate requiered \textwidth
  {\dimexpr\the\paperheight-5.5cm\relax}% calculate requiered \textheight
\recalctypearea

% \begin{figure}[htp] \centering{
%     \includegraphics[scale=1]{zeitplan.pdf}}
%     \caption{Zeitplan IPA}
% \end{figure}

\restoregeometry
\zeitplan
\newpage




\chapter{Arbeitsjournal}
\input{journal/tag1.tex}
\input{journal/tag2.tex}
\input{journal/tag3.tex}
\input{journal/tag4.tex}
\input{journal/tag5.tex}
\input{journal/tag6.tex}
\input{journal/tag7.tex}
\input{journal/tag8.tex}
\section{Tag 9: TODO: Datum}
\begin{table}[H]
    \begin{tabular}{|L{0.4\textwidth}|C{60pt}|C{60pt}|C{60pt}|}
        \hline
        \rowcolor{puzzleblue}\color{white}Tätigkeiten & \color{white}Beteiligte \color{white}Personen & \color{white}Aufwand Geplant (std) & \color{white}Aufwand Effektiv (std) \\
        \hline
        TODO: Tätigkeit & TODO: Beteiligte Personen & TODO: Stunden Soll & TODO: Stunden Ist \\
        \hline
        \textbf{Total} & & TODO: Stunden Soll Total & TODO: Stunden Ist Total \\
        \hline
    \end{tabular}
    \caption{Tätigkeiten Tag 9}
\end{table}

\subsection*{Tagesablauf}


\subsection*{Hilfestellungen}
\begin{itemize}
    \item TODO: Hilfestellungen auflisten
\end{itemize}

\subsection*{Reflexion}
\subsubsection*{Was lief gut}

\subsubsection*{Was lief weniger gut}

\subsubsection*{Meine Erkenntnisse von heute}

\subsection*{Nächste Schritte}

\pagebreak

\input{journal/tag10.tex}

\chapter{Abschlussbericht}

\section{Vergleich Ist/Soll}

\section{Mittelbedarf}

\section{Umsetzungsbericht}

\section{Testbericht}

\section{Fazit zur IPA (Projekt)}

\section{Persönliches Fazit}

\section{Schlussreflexion}

  \part[Projektdokumentation]{Projektdokumentation
                  \begin{center}
                     \begin{minipage}[c]{10.7cm}
                      \small TODO:Name der IPA einfügen \\
                      Autor: TODO: Autor einfügen
                     \end{minipage}
                  \end{center}
                 }
\chapter{Einführung}

\chapter{Analyse}

\section{Ist-Zustand}

\section{Soll-Zustand}

\section{Persönliche Vorgehensziele}

\section{Anforderungen}
\subsection{Nicht funktionale Anforderungen}
 
\subsection{Funktionale Anforderungen}

\section{Abgrenzung}
 
\chapter{Entwurf}

\section{Testkonzept}

% TODO: Testumgebung beschreiben
% Beispiel für Test Tabelle, muess je nachdem angepasst werden
\begin{table}[H]
    \rowcolors{2}{puzzleblue!25}{white}
    \begin{tabular}{|L{0.3\textwidth}|L{0.65\textwidth}|}
        \hline
        \rowcolor{puzzleblue} \multicolumn{2}{|l|}{\color{white}Testfall Nr. 1}  \\[10pt]
        \hline
        \textbf{Testname} &  \\
        \hline
        \textbf{Testmethode} &  \\
        \hline
        \textbf{Voraussetzungen} & \\
        \hline
        \textbf{Testkontext} & \\
        \hline
        \textbf{Erwartetes Resultat} & \\
        \hline
    \end{tabular}
    \caption{Testfall 1}
\end{table}


\chapter{Umsetzung}

\section{Testprotokoll}

% Beispiel für Test Tabelle, muess je nachdem angepasst werden

\begin{table}[H]
    \rowcolors{2}{puzzleblue!25}{white}
    \begin{tabular}{|L{0.3\textwidth}|L{0.65\textwidth}|}
        \hline
        \rowcolor{puzzleblue} \multicolumn{2}{|l|}{\color{white}Resultat Testfall Nr. 1}  \\[10pt]
        \hline
        \textbf{Testname} &  \\
        \hline
        \textbf{Testkontext} &  \\
        \hline
        \textbf{Testperson} &  \\
        \hline
        \textbf{Ausführungs Datum} &  \\
        \hline
        \textbf{Testergebnis} &  \\
        \hline
        \textbf{Beschreibung} &  \\ 
        \hline
        \textbf{Fehlerklasse} & \\ 
        \hline
    \end{tabular}
    \caption{Resultat Testfall 1}
\end{table}


\chapter{Einführung}

\chapter{Sprintabschlüsse}

\section{Abschluss Sprint Initialisierung}

\section{Abschluss Sprint Umsetzung}

\section{Abschluss Sprint Finalisierung}



}
{
  %Tabellen- & Abbildungsverzeichnis als Section und in Inhaltsverzeichnis
  \makeatletter

  \cftsetindents{figure}{0em}{3.5em}
  \cftsetindents{table}{0em}{3.5em}

  \renewcommand\listoftables{%
      \section{\listtablename}%  
      \@mkboth{\MakeUppercase\listtablename}%
              {\MakeUppercase\listtablename}%
      \@starttoc{lot}%
  }

  \renewcommand\listoffigures{%
      \section{\listfigurename}%
      \@mkboth{\MakeUppercase\listfigurename}%
          {\MakeUppercase\listfigurename}%
      \@starttoc{lof}%
  }

  \renewcommand\lstlistoflistings{%
      \section{Code Verzeichnis}%
      \@mkboth{Code Verzeichnis}%
          {Code Verzeichnis}%
      \@starttoc{lol}%
  }
  \makeatother
  
  \part[Anhang und Verzeichnise]{Anhänge und Verzeichnise
                  \begin{center}
                     \begin{minipage}[c]{10.7cm}
                        \small TODO:Name der IPA einfügen \\
                        Autor: TODO: Autor einfügen
                     \end{minipage}
                  \end{center}
                 }

\chapter{Verzeichnise}

\listoftables

\listoffigures

\lstlistoflistings

\renewcommand\bibname{Quellenverzeichnis}
\begin{thebibliography}{9}
    \bibitem[TODO: Name der Quelle]{TODO: Name der Quelle} \url{TODO: URL einfügen}, (TODO: Datum von Tag wo Quelle verwendet wurde)
\end{thebibliography}
\addcontentsline{toc}{subsection}{Quellenverzeichnis}

\chapter{Verwendete Abkürzungen}

\begin{table}[H]
    \rowcolors{2}{puzzleblue!30}{white}
    \begin{tabular}{|L{0.3\textwidth}|L{0.6\textwidth}|}
        \hline
        \rowcolor{puzzleblue} \textbf{\color{white}Abkürzung} & \textbf{\color{white}Bedeutung} \\[12pt]
        \hline
        TODO: Abkürzung & TODO: Beschreibung \\
        \hline
    \end{tabular}
    \caption{Verwendete Abkürzungen}
\end{table}

\chapter{Glossar}

\begin{table}[H]
    \rowcolors{2}{puzzleblue!30}{white}
    \begin{tabular}{|L{0.3\textwidth}|L{0.6\textwidth}|}
        \hline
        \rowcolor{puzzleblue} \textbf{\color{white}Bezeichnung} & \textbf{\color{white}Bedeutung} \\[12pt]
        \hline
        TODO: Wort & TODO: Beschreibung \\
        \hline
    \end{tabular}
    \caption{Glossar}
\end{table}

\chapter{Anhänge}

\section{Sitzungsprotokolle}

\input{code.tex}







}

\end{document}
